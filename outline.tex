\documentclass{article}[12pt]
\usepackage{simplemargins}
\setallmargins{1.0in}
\title{A Framework for the Study of Adaptations \\ Across Culture and Media \\ \small{An Outline}}
\author{\textsc{Akanksha Vyas}}
\date{}
\begin{document}
\maketitle
\begin{itemize}
\item[1. ] Introduction 
\item[2. ] Literature Review \\
This section will study and analysis of existing theories pertaining to film adaptations. Notable theorists include Robert Gidding, Kamilla Elliott, Linda Chair and James Welsh. A detailed analysis of the theories of Linda Hutcheon and a review of adaptation studies in light of video games. 
\item[3. ] The Framework \\
Extending the previous section, this section will clarify the definition of an adaptation and explicate the framework. 
\item[4. ] Evaluation \\
This section will first illustrate examples of adaptations that do not fit the framework and explain why. Tim Burton's Alice in Wonderland based on the book by Lewis Carroll, Disney's Mulan based on the Chinese legend of Hue-Mulan , BBC's Pride and Prejudiced as an adaptation of the book by Jane Austen. \\
Then, the section will discuss examples :of effective adaptations in the scope of our framework. Hitchhikes Guide to the Galaxy Adventure Game by Douglas Adams, based on the radio show by him, and Howl's Moving Castle the Japanese anime and English children's novel. 
\item[5. ] Creation \\
The final part of the paper will discuss the process of creating an adaptation with the aid of this framework. Stating with a Calvin and Hobbes comic strip, we will create a photo essay. 
\item[6. ] Conclusion
\end{itemize}
\end{document}


% LocalWords:  Hutcheon Cartmell Corrigan Whelehman Gidding Kamila Disney's
% LocalWords:  Mulan BBC's anime
