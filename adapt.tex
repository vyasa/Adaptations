% This document is licenced under the Creative Commons
% Attribution-NonCommercial-ShareAlike 3.0 Unported License. With this licence
% you are free to copy, redistribute, transmit and adapt any or all of the work,
% for strictly non-commercial purposes. 
% Go to http://creativecommons.org/licenses/by-nc-sa/3.0/ for more information. \\


\documentclass[12pt]{article}
\usepackage{ifpdf}
\usepackage[utf8]{inputenc}

%Margin - 1 inch on all sides
\usepackage[letterpaper]{geometry}
\geometry{top=1.0in, bottom=1.0in, left=1.0in, right=1.0in}

%Doublespacing
\usepackage{setspace}
\doublespacing

%Setting the font
\usepackage{times}

%Fancy-header package to modify header/page numbering (insert last name)
\usepackage{fancyhdr, graphicx}
\pagestyle{fancy}
\lhead{} 
\chead{} 
\rhead{Vyas \thepage} 
\lfoot{} 
\cfoot{} 
\rfoot{} 
\renewcommand{\headrulewidth}{0pt} 
\renewcommand{\footrulewidth}{0pt} 
%To make sure we actually have header 0.5in away from top edge
%12pt is one-sixth of an inch. Subtract this from 0.5in to get headsep value
\setlength\headsep{0.333in}

%Works cited environment
%(to start, use \begin{workscited...}, each entry preceded by \bibent)
% - from Ryan Alcock's MLA style file
\newcommand{\bibent}{\noindent \hangindent 40pt}
\newenvironment{workscited}{\newpage \begin{center} Works Cited \end{center}}{\newpage }

\begin{document}
\begin{flushleft}
%%%%First page name, class, etc
Akanksha Vyas\\
\today\\
\end{flushleft}
%%%%Title
\begin{center}
\textbf{\large{The Art of Retelling Stories} \\ \small{A Framework to Study TransMedia and Cross-Cultural
Adaptations}} 
\end{center}

%%%%Changes paragraph indentation to 0.5in
\vspace{0.2in}
%\parbox[b]{5.5in}{ This document is licenced under the Creative Commons
% Attribution-NonCommercial-ShareAlike 3.0 Unported License. With this licence
% you are free to copy, redistribute, transmit and adapt any or all of the work,
% for strictly non-commercial purposes. 
%Go to http://creativecommons.org/licenses/by-nc-sa/3.0/ for more information.} 
\setlength{\parindent}{0.5in}

\begin{abstract}
After years of condemnation, when adaptations finally received academic
attention, they were studied a branch of literature or film studies. I believe
that we should expand this scope to understand them across the many different
media and cultures which the modern world has to offer.
This paper hopes to present frameworks to guide the evaluation and creation
of adaptations universal across different media and cultures, prompting
academics to understand them as independent works of art. 
\end{abstract}

Adaptations have faced many challenges in their pursuit for academic
recognition. Though they have been prevalent in popular media 
since as early as the $6^{th}$ century AD, adaptations did not receive academic 
attention till the late 1950's. With \textit{George Blustones's} publication
of `Novel to Film', scholars started to depart from the notion
that adaptations were mere imitations and discussed them as independent works 
of art. Despite this, the study of adaptations we limited to film adaptations of
literature. An increasingly global world with the injection of
many different facets of the \textit{new media} is nudging us to explore adaptations
outside these conventional realms. I believe that the ability to take a story
and present it in a completely different media or culture is an imperative skill in
the modern world. This paper aims to formulate theories for the creation and evaluation of
adaptation that are universal to the field, irrespective of the medium or culture
under consideration.

Despite their limited scope, examining the different adaptation theories 
constructed over the last five decades reveals many fundamental ideas about
adaptations. These ideas can be generalized to a broader scope,
yielding techniques that range across, not only different media, but also across cultures.
I begin with an analysis of the existing adaptation theories and present a
precise definition of an adaptation. Using this definition, I construct a
framework which hopes to guide the evaluation of adaptations. Two case studies
illustrate this framework and reveal techniques about the creative process.
Using these techniques to extend the definition established previously, I
present a framework to guide the creation of adaptations. A creative process
evolved from this framework is described to provide a better understanding of
it. 

Before the 1960's, adaptations were condemned for their mimetic nature. The
opinion that film adaptations of literature lacked creativity and expression
was shared by writers and film-makers alike. Modernist writer Virginia Woolf
believed that the film medium could never capture the rich imagery and dense
characterization that constitutes great literature. Similarly,
twentieth century French director Alain Resnais believed that writers completely express
themselves with through work, leaving nothing for the film-makers to do
(Beja 79).

In his famous 1967 publication, George Bluestone discussed techniques 
for the analysis of adaptations, criticising film reviewers for
evaluating film adaptations based on their fidelity to the original work. He
reckoned that such analysis techniques were purely quantitative and did not
address the adaptation process. `They merely establish the fact of reciprocity; they
do not indicate its implications for aesthetics (Bluestone 5).' Additionally, they
advanced the idea `that the novel is a norm and the film deviates at its peril
(5),' thus classifying film adaptation as an inferior art form. Bluestone 
reproached such theories, arguing that `the filmed novel, in spite
of certain resemblances, will inevitably become a different artistic entity
   from the novel on which it is based (64),' and should be analysed in that
   capacity.

Despite his pioneering work in the field of adaptation theory, Bluestone did not
address the actual process of adapting a work from literature to film. In the
1980's film critic, writer and poet B\'{e}la Bal\'{a}zs approached this issue.
Bal\'{a}zs argued that the original text should merely be raw material to the
film-maker, and its form should be ignored. An adaptation should be a
\textit{reinterpretation} of the original work. The creative process was further discussed in \textit{Film and Literature},
where Morris Beja raised the question of fidelity in an interesting manner. `What
should be uppermost in a film-maker's mind: the integrity of the original work,
or the integrity of film to be based on the work? (Beja 80)'. Beja contrasted Bal\'{a}zs'
argument with that of the French film critic Ande\'{e} Bazin. Bazin asserted
films which focus too much on being cinematic tend to become
copies of each other. He contended that film-makers should `care the least for
fidelity in the name of the so-called demands of the screen who betray at one
and the same time both literature and the cinema (81).' After studying various theories
of adaptation, Beja concluded that although it is important to
consider what the adapted film takes from the original text, it is equally
important to consider what the film brings to the original text. The adapted
work should not be an illustration nor a departure, it should be `a work of art
that relates to the book from which it derives, yet [is] also independent (88).' 

In 1999 Linda Cahir presented an aesthetic rubric to guide creators of
film translations of literature. According to
her rubric, an adaptation should be the film-makers interpretation of the integral 
meaning of the literary text. It should be a collaboration of film-making skills
resulting in self-reliant work that is an aesthetic offspring, but not completely
independent of the source material.

Into the late twentieth century, adaptations were studied as a branch of
either literature or film studies. Although theories had started to depart from the
course of fidelity-based discussions, the source text was still considered a critical part
of the creative and analytical process. In the early twenty-first century,
work by Linda Hutcheon and Thomas Leith redefined this norm. 
They proposed a paradigm to realize adaptations as being `foundational to the 
extent that any audience experiences an adaptation as an adaptation (Leith 177).' Hutcheon
contrasted parodies with museum exhibits to illustrate how
audiences view the former as an adaptation, but not the latter. With this she
exemplifies her notion that audiences `must experience the 
adaptation \textit{as an adaptation} (Hutcheon 172).'

Hutcheon was the first scholar to go beyond the scope of literature into film, 
exploring adaptations across different forms of media. 
Her 2006 publication of \textit{A Theory of Adaptations} played a significant 
role in establishing adaptation theory as an academically recognized field of
study. Despite her revolutionary work in adaptation theory, much of
Hutcheon's work is in unison with that of her contemporaries. She 
suggested a method `to think about unsuccessful adaptations \textellipsis not 
in terms of infidelity to the prior text, but in terms of a lack of
creativity and skill to make the text one's own and thus autonomous (Hutcheon
20).'

Before Hutcheon and Leith entered the colloquy, theorists considered 
the source of an adapted work to be a critical part of the discussion.
Reproaching this entirely, Hutcheon and Leith required that an adapted work must
be viewed as an adaptation by the audience. I argue that both these
definitions subordinate adaptations, denying them the individuality they
deserve.  An adaptation is the product of a creative process, and is as much 
a work of art as the original. When defining adaptations, I think it might
be useful to shift focus to a common thread that appears across the theories 
discussed. Theorists agree that adaptations 
need to be new, different from the source text and independent in their 
own capacity. Hutcheon determined that the role of an adapter is to 
reinterpret the source text (20). The media and cultural paradigm of the adapted 
work influences this interpretation. Analogous to languages, each medium is
governed by grammatical rules that differentiate it from other media.
Discussion of the benefits and flaws of literary and
film media has been a pertinent part of adaptation theory in the twentieth century.
In 2008 Moore illustrated the differences between
the grammar of the new media and that of traditional media. 
Similarly, different cultures have unique traditions and priorities
that characterise them. Culture can be different in ethnicity,
geography, time, or any system governed by a different heritage. 
An understanding of these rules and traditions helps 
adapters use the destination medium and/or culture to its full potential. 

Adaptation theorists take for granted that the adapted work must resemble its
source. The challenge is to determine what must be preserved. Russian theorist 
Tomeshavsky understood the story to be the irreducible part of any work,
serving as the pretext to the plot. According to him, the plot `is where the 
artistry lies (Lemon and Reis 61)'. Similarly, Kamilla Elliot distinguishes between content and
form. In \textit{Rethinking the Novel Film Debate}, she discusses how 
content can be extracted from the literary form and transferred to the 
form of film (Elliot 133). Other measures of the spirit of a work include 
Linden's understanding that the tone is central or Seger's theory that
the style of any work is at its core (Giddings).
Each of these formalisations come together to constitute the irreducible part,
or more concisely, the \textit{spirit} of a work. Revisiting Hutcheon's
conception that it is the job of an adapter to interpret the source text, we can merge
these ideas. Adaptations contain the adapter's interpretation of the story, 
content, tone, style or anything that constitutes the core of the source work. 
Sequence of events, or even individual events, are inconsequential in the
adaptation, provided the dominant themes are reflected. The
challenge of good adaptations is to ensure that the interpretations of the
source work captures its central ideas. 

An adaptation can be defined as a 
work of art, created as an interpretation of the source to a new media and/or 
cultural paradigm. When creating an adaptation, the artists should focus foremost
on interpreting the source and making it their own. Following this, they
should identify and understand the media and cultural domain of
the adapted work to effectively bring their adaptation to life in that domain.
Evaluating adaptations differs from their creation because the focus should 
be on the affective translation into the new culture and media, and not on the
interpretation. Criticising the artist on their interpretation of the source
would require the evaluator to have a complete understanding of the creative 
process, which would be impractical at best. It is fair to judge painters on
their technique but presumptions to think that anyone other than the artist can
truly understand their interpretation of nature. Talking this as an analogy,
evaluators should focus on the technique of the artist, not their
interpretation. The next section expands this definition into a framework
for the evaluation of adaptations.

\pagebreak
\subsection*{Evaluating Adaptations}
An adaptation is the artistry of bringing an existing work to life in a new
media or cultural domain, and should be appraised accordingly. An evaluator
of an adapted work should consider the following framework.

\begin{enumerate}
\item The adapted work utilizes the grammar of the media in which it is set.
\item The adapted work accords to the culture it represents.
\end{enumerate}

To understand this framework better, each criteria is illustrated
through case studies. In addition to exemplifying the framework, 
each example provides useful insights into the creative process. 
These insights will play a pivotal role while constructing a
framework for creating adaptations.  

J.K. Rowling's 1997 fantasy novel, \textit{Harry Potter and the Sorcerer's Stone} 
was adapted to film 2001. Though the film was immensely popular, it has been
criticized for its lack of cinematic dimension (Cartmell and Whelehan 37).
Analyzing this adaptation with regard to the grammar of the film medium
avails an understanding of adaptations that do not exploit the potential of the
destination medium.

The grammar of cinema differs from that of the literary medium because it
creates `a perfect illusion of [the] reality (Giddings)' which the literature describes. 
Rowling presents her reader with two different realities. 
The first is very ordinary, and the second is of a world of magic. 
The book perfectly juxtaposes the eccentricity of the fantasy world with the
mundane nature of the ordinary world by contrasting strong imagery with words
of a dull overtone. However, despite the visual techniques the medium had to offer, the
film adaptation failed to bring the imagery of the fantasy world to life.

Fantasy films of a similar flavour commonly contrast rich colours and bright lighting
with their bland counterparts to bring out the allure of the fantasy world.
Figure 1  and 2 compare the differences between the two realities in \textit{Harry
Potter and the Sorcerer's Stone} to that of \textit{The Wizard of Oz}. The bright, 
rich colours in the magical world of Oz emphasize it against
its counterpart. In contrast, the two realities in the Harry Potter film seem
indistinguishable. 

\begin{figure}
\begin{minipage}[b]{0.5\linewidth}
\centering
\includegraphics[scale=0.3]{oz.jpg}
\caption{The Wizard of Oz. Top: Dorothy near her home in Kansas. Bottom: The World of Oz. Photo Credit: dummidumbwit.files.wordpress.com}
\label{fig}
\end{minipage}
\hspace{0.5cm}
\begin{minipage}[b]{0.5\linewidth}
\centering
\includegraphics[scale=0.3]{hp.jpg}
\caption{Harry Potter. Top: Harry with the Dursley's at Private Drive. Bottom:
Happy and Ron at Hogwarts. Photo Credit: gabtor.files.wordpress.com}
\label{fig:figure2}
\end{minipage}
\end{figure}

Cartmell and Whelehan argue that original novel conforms to the film medium more
precisely than its film counterpart. Thought the notion of deeming a film less 
cinematic than a novel seems paradoxical, a closer understanding of the texts 
substantiates their argument. A study of the \textit{ Sorting Hat }
sequence provides a strong example. Rowling's description of `The Sorting Hat sequence 
borrows from the techniques of flash cutting (40)', Shot-Reverse-Shot technique and 
describes a close-up of Harry and the hat.  In the film, this scene is shot with 
two wide angle shots that swap once.

In its attempt to capture every detail presented in the book, the film missed 
many cinematic opportunities. It
is important for creators to understand that not every detail of the source can
be adapted, and focus on extracting the spirit of the source while
keeping the grammar of the destination medium in mind. 

Shakespeare's \textit{Macbeth} has always been a popular source texts for film
adaptations. A 2006 film adaptation of the play, \textit{Maqbool}, was set in
the Mumbai underworlds of India.
Directer Vishal Bharadwaj changed many plot elements and secondary
characters to successfully transform the film into the domain of its destination
culture. Two of the major changes are discussed below.

Though Indian culture recognizes witches as historical myths,
they are not prevalent in modern society. Bharadwaj replaces the three witches
with two police officers who are working for the mafia. One of the officers
fancies himself a fortune teller and makes prophecies that are usually to his
advantage. The two officers meddle with events to make these prophecies come true,
removing the element of supernatural from the plot.

The presence of romance is convention in \textit{Bollywood} cinema, and the
audience anticipates it. The relationship between Macbeth and Lady Macbeth is
central to the original text, but its lacks romantic conflict. In the film, 
\textit{Nimmi} (Lady Macbeth) is the mistress of the King. She falls
in love with \textit{Maqbool} (Macbeth) and he returns her affection. 
Nimmi's thirst for power conflicts with her lust, and she encourages 
Maqbool to kill the King. This accentuates their 
relationship with an added element of conflict. Maqbool's
victory would not only bring him power, but also the women he loves. 

Bharadwaj stripped the original play of 
any elements that did not align with the chronological or geographic culture of 
the destination medium, keeping only the elements that were central to the story.

\pagebreak

\subsection*{Creating Adaptations}
Before we actualize a framework for the creating of adaptations, let us review the
case studies to understand techniques that could be beneficial to the creative
process. The \textit{Harry Potter} case study emphasized the importance of
letting the domain of an adaptation influence the interpretation of the source
work. It also promoted the fact that, to be able to use it to its maximum
potential, a concrete understanding of this domain is crucial. The
\textit{Macbeth} case study presented a convincing argument towards getting rid
of story components to preserve authenticity.

Integrating these with the definition of adaptations discussed previously, a
framework to guide the creation of adaptations is presented below.
\begin{itemize}
\item Identify the media and cultural domain of the adapted work.
\item With an understanding of the domain of the adaptation, identify the
central ideas of the source work, interpreting it to make it your own.
\item Remove all elements of the source culture that would not seem natural in
the domain of the destination culture. It is imperative to ensure that the core
ideas of the story are not lost in doing so.
\item With an understanding of the grammar of the destination media, produce 
the adaptation. This could be the act of writing a book, directing a film ect.
\end{itemize}

To understand this framework better, I put myself in the shoes of the artist and
used it as a guide to create an adaptation. The creative process is described
below with the hope that it will present a better understanding of this framework.

American cartoonist \textit{Bill Watterson} wrote and illustrated the `Calvin
and Hobbes' comic strip between 1985 and 1995. Though the comic consists of numerous
hilarious anecdotes about the antics of 6 year old Calvin and his stuffed toy
tiger Hobbes, it subtly address the loneliness that causes Calvin to find
comfort with an imaginary friend.

I was inspired by a particular strip of this comic titled \textit{The Raccoon Story},
because it is different from many other `Calvin and
Hobbes' strips in that it presents a positive relationship between Calvin and
his parents. The comic series is usually characterized by Calvin and his parents in 
constant feud with each other. In this particular strip, Calvin and Hobbes discover an
injured raccoon. Calvin asks his mother for help, and the entire family tries to
nurse the raccoon back to health. Unfortunately, their efforts are in vain 
and the raccoon dies, but the circumstance brings the family together. Calvin is
distraught by the event, as he starts to understand the inevitability of death.

The first stage of the creative process was to identify the culture and medium of
the adaptation. I collaborated with amateur photographer Gaurav Swan, and we
decided to create a photo-story set in urban India. We chose to set the adaptation in India
because we both grew up there, and were well versed with its cultural
intricacies. Gaurav understood photography techniques, and I studied the works 
of Ryan Contello to learn about photo-stories. 

The next stage was to interpret the comic strip. To us, it told the story 
of how death of a stray brought an isolated family together. It also reflected on
of how a small child struggled with the death of a loved one. After
considering both interpretations, we decided to focus on the first one. 
After considering the gramatic features of photo-stories, we concluded that 
internal conflict is hard to portray without any 
form of text or sound. Additionally, India has been witness to a demographic
chance, where the culture has moved from that of large extended 
families, to one of small nuclear families. This story touches upon the
isolation and disconnect that many children in such families often experience.

It was now time to remove all elements that pertain to
the source culture, without sacrificing the spirit of the story. The original
story is set in the suburban United States in the $20^{th}$ century,
characterized by big houses, open grounds, and a raccoon (an animal which is
native to North America). The adapted work is set in a small apartment building in
urban India. The open grounds were replaced by a small playground and the
racoon by a kitten, common to many urban cities in India.

The last stage of the project was production. A noteworthy challenge faced
was that we could not find a stuffed toy tiger. To overcome
this, we decided to replace Hobbes with a stuffed toy caterpillar. The rest of the production
was quite smooth. We actualized a 11 frame photo-story about how the
death of a stray cat brought an isolated family closer together as the parents helped their
son cope with the loss.

Academic theories of adaptations advocate the recognition of adaptations as
independent works of art. The two frameworks presented in this paper
provide evaluators and creators of adaptations with guidelines to realize
adaptations in this manner. Not limited by media or cultural
restrictions, these guidelines can be applied to a broad scope, catering to a
global world with a variety of different media.


\begin{workscited}
Hutcheon, Linda. \textit{A Theory of Adaptation}. New York:Routledge. 2006. Print.\\
Miller, Carolyn Handler. \textit{Digital Storytelling: a Creator's Guide to Interactive
Entertainment}. Focal Press. 2004. Print.\\
Leitch, Thomas. ``Adaptation, the Genre". \textit{Adaptation}. 1. 2. (2008):
106-120. Print. \\
Moore, Michael Ryan. ``Adaptation and New Media". \textit{Adaptation}. 3. 2. (2008):
192-197. Print. \\
Elliott, Kamilla. \textit{Rethinking the Novel Film Debate}. Cambridge:Cambridge
University Press. 2003. Print.\\
Cahir, Linda Costanzo. \textit{Literature into Film : Theory and Practical
Approaches}. Jefferson:MacFarland. 2006. Print. \\
Bluestone, George. \textit{Novels into Film}. California:University of California Press.
1961. Print.\\
Beja, Morris. \textit{Literature and Film, an Introduction}. New York:Logman. 1979.
Print. \\
Lemon, Lee T and Reis, Marion J. \textit{Russian Formalist Criticism: Four
Essays}. Nebraska:University of Nebraska Press. 1965. Print.\\
Giddings, Robert. \textit{Screening the novel : the theory and practice of literary
dramatization}. New York:St. Martin's Press. 1990. Print \\
Cartmell and Whelehan. \textit{Books in Motion}. New York:University of Wales. 2005. Print.
\end{workscited}
\end{document}

