% This document is licenced under the Creative Commons
% Attribution-NonCommercial-ShareAlike 3.0 Unported License. With this licence
% you are free to copy, redistribute, transmit and adapt any or all of the work,
% for strictly non-commercial purposes. 
% Go to http://creativecommons.org/licenses/by-nc-sa/3.0/ for more information. \\


\documentclass[12pt]{article}
\usepackage{ifpdf}
\usepackage{simplemargins}
\usepackage{setspace}
\usepackage[utf8]{inputenc}
\usepackage{cite}
\doublespacing
\setallmargins{1.2in}
%
%Works cited environment
%(to start, use \begin{workscited...}, each entry preceded by \bibent)
% - from Ryan Alcock's MLA style file
%
\newcommand{\bibent}{\noindent \hangindent 40pt}
\newenvironment{workscited}{\newpage \begin{center} Works Cited
\end{center}}{\newpage }

\setlength{\textwidth}{5.8in} % set width of text portion
%\renewcommand{\headrulewidth}{12pt} 
%\renewcommand{\footrulewidth}{0pt} 
%To make sure we actually have header 0.5in away from top edge
%12pt is one-sixth of an inch. Subtract this from 0.5in to get headsep value
%\setlength\headsep{0.333in}
\begin{document}
\begin{flushleft}
%%%%First page name, class, etc
Akanksha Vyas\\
\today\\
\end{flushleft}
%%%%Title
\begin{center}
\textbf{\large{} \\ \small{A Framework to Study Trans-Media and Cross-Cultural
Adaptations }} 
\end{center}

%%%%Changes paragraph indentation to 0.5in
\vspace{0.2in}
\parbox[b]{5.5in}{ This document is licenced under the Creative Commons
 Attribution-NonCommercial-ShareAlike 3.0 Unported License. With this licence
 you are free to copy, redistribute, transmit and adapt any or all of the work,
 for strictly non-commercial purposes. 
Go to http://creativecommons.org/licenses/by-nc-sa/3.0/ for more information.} 
\setlength{\parindent}{0.5in}
\subsubsection*{Literature Review}
Before the 1960's, adaptations were condemned for their mimetic nature. The
opinion that film adaptations of literature lacked creativity and expression
was shared by writers and film makers alike. Modernist writer Virginia Woolf,
believed that the film medium could never capture the rich imagery and dense
characterization that constitutes great literature. Similarly, Alain Resnais,
twentieth century French director, believed that writers completely expresses
themselves with their work, leaving nothing for the film makers to do
(Beja 79).

The 1960's brought a positive change to the study of adaptations. The field
gained academic attention with George Bluestone's 1957 publication of \textit{Novel to Film}. 
Bluestone discussed techniques for the analysis of adaptations, criticising film reviewers for
valuating film adaptations based on their fidelity to the original work. He
reckoned that such analysis techniques were purely quantitative and did not
address the adaptation process. `They merely establish the fact of reciprocity;they
do not indicate its implications for aesthetics'(Bluestone 5). Additionally, they
advanced the idea `that the novel is a norm and the film deviates at its peril'(5), 
thus classifying film adaptation an inferior art form. Bluestone 
reproached such theories, arguing that `the filmed novel, in spite
of certain resemblances, will inevitably become a different artistic entity
   from the novel on which it is based'(64), and should be analysed in that
   capacity.

Despite his pioneering work in the field of Adaptation Theory, Bluestone did not
address the process of adapting a work from literature to film. Two decades
later; film critic, writer and poet B\'{e}la Bal\'{a}zs approached this issue.
Bal\'{a}zs argued that the original text should merely be raw material to the film
maker, and its form should be ignored. An adapted work should be a
`re-interpretation' of the original work. 

In \textit{Film and Literature}, Morris Beja raises the question of fidelity in an interesting form, `What
should be uppermost in a filmmakers mind: the integrity of the original work,
or the integrity of film to be based on the work'(Beja 80). Beja contrasts Bal\'{a}zs'
argument with that of the French film critic, Ande\'{e} Bazin. Bazin argued
films that focus too much on being cinematic tend to become
copies of each other. He contended that film makers should `care the least for
fidelity in the name of the so-called demands of the screen who betray at one
and the same time both literature and the cinema'(81). Studying the various theories
on the adaptation process, Beja concluded that although it is important to
consider what the adapted film takes from the original text, it is equally
important to consider what the film brings to the original text. The adapted
work should not be an illustration nor a departure, it should be `a work of art
that relates to the book from which it derives, yet [is] also independent'(88). 

Up into the late twentieth century, adaptations were studied as a branch of
literature or film studies. Though theories had started to depart from the
course of fidelity based discussion, the source text was still considered a critical part
of the creative and analytical process. In the early twenty-first century,
work by Linda Hutcheon and Thomas Leith redefined this norm. 
They proposed a paradigm to realize adaptations as being `foundational to the 
extent that any audience experiences an adaptation as an adaptation'(Leith). Hutcheon
contrasted parodies with museum exhibits to illustrate how
audiences view the former as an adaptation, but not the latter. With this she
exemplifies her notion that audiences `must experience the 
adaptation \textit{as an adaptation}'(Hutcheon 172).

Despite her revolutionary work in adaptations, much of
Hutcheon's work is in unison with that of her contemporaries. In 
\textit{A Theory of Adaptations} she discusses both the analytical and 
creative process. `Perhaps one way to think about unsuccessful adaptations
is not in terms of infidelity to the prior text, but in terms of a lack of
creativity and skill to make the text one's own and thus autonomous'(20). She
understands the creative process to be `of interpreting and then creating
something new'(20).

Hutcheon was the first scholar to go beyond the scope of literary and film 
adaptations, exploring adaptations across different forms of media. 
Her 2006 publication of \textit{A Theory of Adaptations} played a significant 
role in establishing Adaptation Theory as an academically recognized field of
study. The launch of Oxford University Press Journal, titled \textit{Adaptation},
actualized this. Thought it was established to `interrogate the phenomenon
of literature on screen from both a literary and film studies perspective', 
the journal has provided forum for discussion beyond those media paradigms.
The popularity of the \textit{new media} inspired discussion of `technical
restraints and social practices'(Moore) that govern a variety of different media. 

Hutcheon and Leith's definition necessitates that an adapted work is
viewed as adaptations by the audience, deeming them to a subordinate
degree. Theorist Robert Stam argues that `literature will always have axiomatic
superiority over any adaptation of it because of its seniority as an art from'
(Hutcheon 4). Both these definitions deny adaptations the individuality
they deserve. An adaptation is the product of a 
creative process, and is as much a work of art as the original. 

While defining adaptations, it might be useful to shift focus to a common thread that
appears in many adaptation theories throughout history. Theorists agree that adaptations 
need to be new and different from the source
text, independent in their own capacity. Hutcheon determined that the role of an adapter is to 
reinterpret the source text. The media and cultural paradigm of the adapted 
work influences this interpretation. Analogous to languages, each medium is
governed by grammatical rules that differentiate it from other media. These
rules help understand the difference between \textit{cinematic} versus
\textit{literary} work. Consideration of the benefits and flaws of literary and
film media has been a pertinent part of Adaptation Theory in the twentieth century.
Moore's discussion of video games in 2008 illustrated the differences between
the grammar of the \textit{new media} in comparison with that of traditional media. 
Similarly, different cultures have unique traditions and priorities
that characterise them. An understanding of these rules and traditions helps 
adapters use the destination media and/or culture to its full potential. 

Adaptations theorists take for granted that the adapted work must resemble its
source material. The challenge is to determine what must be preserved. Russian
theorist Tomeshavsky defines the `irreducible part of the work' as being its
motif. Though most theories consider the story to be `the core of what is transposed
across different media', film reviewers and the audience members `experience the
story in a particular material form', inseparable from the mode of mediation.
Tomeshavsky understands the story to be the core, serving as the pretext to the
plot. According to his, the plot `is where the artistry lies' and is 
`is organized by the artist into a sequence to suit
   his own purposes'. Other measures of the \textit{spirit} of a work include
   Lindens understanding that the tone is central, or Seger's theory that
   the style is the irreducible part of a work.

Each of these formularization come together to constitute the spirit of a work.
Tomeshavsky's definitions of story, congruent with Elliots definition of
content, is central to a work and thus irreducible. The plot or form may be omitted by the
adapter. Sequence of events, or even individual events, are inconsequential in the
adaptation provided the dominant themes are present. The importance of
preserving the tone and style is more subjective. There are some stories where
tone and style is central to the impact of the work, where others focus more on
content. The writings of P.G. Woodhouse and Oscar Wilde rely entirely on the style, where as
style is inconsequential in adapted works of Jane Austen. It is the discretion of
the adapter to access the pertinency of the style or tone of the source material.

An adaptation can be defined as a work of art, void of all elements specific to
the source media and culture, that transports the spirit of the work to a new
media and/or cultural paradigm. When a work borrows some themes from its source
without capturing the spirit, it is a mere inspration, whereas, work that
preserves the media and culural conventions of the source material along the story should
be regarded as a copy. Similerly, adaptations set in the same medium and culture
as their source should be regarded as copies, as the follow the save conventions
as their source. Here culture can be defined to mean a different enthicity,
geography, time, or any system governed by a different heretage. 
It the the task of an adapter to find the balance between
these two realms.
\pagebreak
\subsubsection*{Framework}

In \textit{Literature into Film: Theory and Practical Approaches}, Linda Cahir 
presented an aesthetic rubric for film translations of literature. According to
her rubric, an adaptation should be the filmmakers interpretation of the integral 
meaning of the litrary text. It should be a collabration of filmmaking skills
resulting in self-relient work that is an asthetic offspring, but not completely
independednt of the source material.


Using our definition of adaptations and borrowing from Cahir's rubric, we can
create a framework to address adaptations across media and cultural paradigms. 
There are three critical points that should be considered while creating and
valuating adaptations. 

\begin{enumerate}
\item The adapted work must preserve the spirit of the source material.
\item The adapted work must be set in a medium and/or culture that is different
from its source.
\item The adapted work must compass expressions of the destination media and/or
culture, as the case may be.
\end{enumerate}

To understand this framework better, the rest of the paper will focus on
illustrating the framework through examples.
The following table describes a summary of the adaptations discussed.

\pagebreak
\subsubsection*{Good}
\pagebreak
KE : content and form
Contemporaries 
Stam 

\end{document}
